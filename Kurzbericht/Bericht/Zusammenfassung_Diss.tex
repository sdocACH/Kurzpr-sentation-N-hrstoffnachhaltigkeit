\begin{otherlanguage}{ngerman}
\chapter*{Zusammenfassung}
\label{chap:Zusammenfassung}
\addcontentsline{toc}{chapter}{Zusammenfassung}

Das Ziel dieser Dissertation war die Entwicklung von Methoden, welche den Wert von grossräumigen Waldinventuren steigern, indem diese auch für Schätzungen auf deutlich kleinräumigeren forstlichen Behandlungseinheiten verwendet werden können. In diesem Rahmen wurden zwei Ansätze getestet, welche beide auf der Kombination von terrestrisch erhobenen Stichprobendaten mit flächendeckend verfügbaren Fernerkundungsdaten beruhen. Hinsichtlich einer zukünftigen Anwendung auf operationeller Ebene wurden die Verfahren anhand der Verlässlichkeit der Genauigkeitsangaben für ihre Schätzungen, der Grössenordnung der erreichbaren Schätzgenauigkeiten, sowie ihrer Eignung zur Anwendung über grosse Flächen bewertet.\par

Der erste Ansatz bestand aus der Anwendung design-basierter Regressionsschätzer für Kleingebietsschätzungen und wurde in einer Studie im Bundesland Rheinland-Pfalz (Deutschland) getestet. Zunächst wurde das bestehende Netz der Nationalen Waldinventur zu einem zweiphasigen Stichprobendesign erweitert. Anschliessend wurden durch die Kombination mit landesweit verfügbaren Fernerkundungsdaten Schätzungen der Holzvorräte auf Forstamt- und Forstrevierebene berechnet und die erreichten Schätzgenauigkeiten evaluiert. Die Ergebnisse der Studie werden in drei aufeinanderfolgenden Kapitel vorgestellt: Das erste Kapitel gibt einen umfassenden Überblick über design-basierte Regressionsschätzer mit speziellem Fokus auf Kleingebietsschätzungen und erläutert deren Implementierung in eine Statistik-Software. Im zweiten Kapitel wird auf drei Herausforderungen bei der Regressionsmodellierung eingegangen. Hierzu zählen der zeitliche Versatz von terrestrischen Aufnahmen und Erhebung der Fernerkundungsdaten, die optimale Ableitung erklärender Variablen unter unbekannter räumlicher Ausdehnung der terrestrisch durchgeführten Aufnahme (Winkelzählprobe), und die Verwendung von Baumarteninformation aus Satellitenbildklassifikation als zusätzliche erklärende Variable. Das dritte Kapitel illustriert die Durchführung der Kleingebietsschätzungen als Synthese aus Kapitel 1 und 2, und präsentiert einen Vergleich der erzielten Schätzgenauigkeiten mit jenen, welche sich unter ausschliesslicher Benutzung der terrestrischen Inventurdaten (einphasige Schätzung) ergeben. Durch die Anwendung der Kleingebietsschätzer konnte die Varianz der einphasigen Schätzung im Mittel um 43\% und 25\% auf Forstamt- und Revierebene reduziert werden, was zu mittleren Schätzfehlern von 5\% und 11\% führte. Damit untermauern diese Resultate das grosse Potential von design-basierten Kleingebietsschätzern für die Verwendung von Nationalinventurdaten auf kleinräumigen forstlichen Managementebenen. Zudem zeigte sich, dass Zeitdifferenzen zwischen terrestrischen- und Fernerkundungsdaten zu einer Verschlechterung der Modellgenauigkeit führen. Dieser Effekt konnte jedoch durch die Verwendung des Aufnahmedatums als kategorielle Variable im Regressionsmodell deutlich verringert werden. Auch führte neben der Vegetationshöheninformation die zusätzliche Verwendung der geschätzten Hauptbaumart eines Stichprobenpunktes zu einer Verbesserung der Modellgenauigkeit. Durch die Anwendung eines Kalibrierungsmodelles war es zudem möglich, die Effekte von Fehlklassifikationen in der Baumartenkarte auf das Regressionsmodell zu neutralisieren. Die Einbeziehung der kategoriellen erklärenden Variablen im Regressionsmodell erweiterte dabei den gewählten Ansatz zu Post-Stratifizierung, welches ein bekanntes und effizientes Mittel zur Varianzreduktion von Schätzungen darstellt.\par

Der zweite Ansatz Bestand aus der Anwendung eines modelabhängigen Mapping-Verfahrens, in welchem der Flächenbezug der Modellprognosen (Pixel der Vorhersagekarte) der räumlichen Ausdehnung eines Stichprobenkreises entspricht. Damit solche räumlich hochaufgelösten Prognosekarten zur Lokation von forstlichen Eingriffen auf kleinräumigsten Ebenen benutzt werden können, ist der Anspruch an ihre Genauigkeit hoch. Daher war es Ziel, für Prognosekarten kontinuierlicher Waldattribute eine zusätzliche Methode zur Bereitstellung der Genauigkeit pro Kartenpixel zu testen. Dies wurde durch die Berechnung der Nutzergenauigkeit, welche für die Genauigkeitsanalyse von Klassifikationen angewendet wird, für definierte Prognoseintervalle realisiert. In diesem Rahmen wurde auch ein Optimierungsalgorithmus getestet, welcher die Spannweite der Prognosen automatisch in kleinstmögliche Intervalle mit grösstmöglichen Nutzergenauigkeiten einteilt. Der Ansatz wurde in einem subalpinen Testgebiet in der Schweiz am Beispiel einer Holzvorratskarte getestet. Daten der kantonalen Waldinventur dienten dabei als Referenzdaten um die Genauigkeit der Prognosekarte zu validieren. Die Ergebnisse deuten an, dass die Vorhersagefehler einzelner Wertebereiche oft deutlich höher sind als auf Basis von generalisierenden Kriterien der Modellgenauigkeit vermutet werden kann. Die vorgeschlagene Validierungsmethode bietet daher eine zusätzliche Genauigkeitsangabe, welche einfach berechnet werden kann und die Information bezüglich der Kartengenauigkeit verbessert. Die Optimierungsmethode kann dazu benutzt werden, eine Prognosekarte in Wertebereiche variabler Länge einzuteilen, welche die zugrundeliegende Modellgenauigkeit bestmöglich reflektieren.

\end{otherlanguage}



